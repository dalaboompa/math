\documentclass{beamer}
\usetheme{Boadilla}
\usepackage{algorithm,algorithmic}
\renewcommand{\mod}{\textbf{ mod }}
\renewcommand{\div}{\textbf{ div }}
\renewcommand{\implies}{\rightarrow}
%Information to be included in the title page:
\title{COMPSCI 1DM3 Tut 4}
\subtitle{Week 08}
\author{Zitong Gu}
\institute{McMaster University}
\date{Winter 2023}

\begin{document}

\frame{\titlepage}

\begin{frame}[t]{Worst-Case Complexity}
    \begin{example}
        Give a big-\textit{O} estimate for the number of operations (where an operation is an addition or a multiplication) used in this segment of an algorithm. \begin{algorithm}[H]
            \begin{algorithmic}[1]
                \STATE $t := 0$
                \FOR{$i=1$ to $3$}
                \FOR{$j=1$ to $4$}
                    \STATE $t := t + ij$
                \ENDFOR
                \ENDFOR
            \end{algorithmic}
        \caption{pseudocode}
        \end{algorithm}
        \vspace{2mm}
    \end{example}
\end{frame}

\begin{frame}[t]{Worst-Case Complexity}
    \begin{example}
        Give a big-\textit{O} estimate for the number of operations (where an operation is an addition or a multiplication) used in this segment of an algorithm. \begin{algorithm}[H]
            \begin{algorithmic}[1]
                \STATE $t := 0$
                \FOR{$i=1$ to $n$}
                \FOR{$j=1$ to $n$}
                    \STATE $t := t + i + j$
                \ENDFOR
                \ENDFOR
            \end{algorithmic}
        \caption{pseudocode}
        \end{algorithm}
        \vspace{2mm}
    \end{example}
\end{frame}

\begin{frame}[t]{Complexity Analysis}
    \begin{example}
        What is the largest $n$ for which one can solve within one second a problem using an algorithm that requires $f(n)$ bit operations, where each bit operation is carried out in $10^{-9}$ seconds, with these functions $f(n)$? \begin{enumerate}
            \item $\log{n}$
            \item $n$
            \item $n\log{n}$
            \item $n^2$
            \item $2^n$
            \item $n!$
        \end{enumerate}
    \end{example}
\end{frame}

\begin{frame}{Complexity Analysis - Insertion Sort (not on test or exam)}
    \begin{example}
        \begin{algorithm}[H]
            \caption{Insertion-Sort($A$, $n$)}
            \begin{algorithmic}[1]
                \FOR{$i = 2$ to $n$}
                      \STATE $key = A[i]$
                      \STATE $j = i - 1$
                      \WHILE{$j > 0$ and $A[i] > key$}
                        \STATE $A[j+1] = A[j]$
                        \STATE $j = j -1$
                      \ENDWHILE
                      \STATE $A[j+1] = key$
                \ENDFOR
            \end{algorithmic}
        \end{algorithm}
        \vspace{2mm}
    \end{example}
\end{frame}

\begin{frame}[t]{Divisibility}
    \begin{example}
        Show that if $a \mid b$ and $b \mid a$, where $a,b \in \mathbb{Z}$, then $a = b$ or $a = -b$. 
    \end{example}
\end{frame}

\begin{frame}[t]{Divisibility}
    \begin{example}
        Show that if $a,b,c,d \in \mathbb{Z}$ with $a \neq 0$ such that $a \mid c$ and $b \mid d$, then $ab \mid cd$. 
    \end{example}
\end{frame}

\begin{frame}[t]{Divisibility}
    \begin{example}
        Show that if $a,b,c \in \mathbb{Z}$ with $a \neq 0$ and $c \neq 0$ such that $ac \mid bc$, then $a \mid b$.
    \end{example}
\end{frame}

\begin{frame}[t]{Divisibility}
    \begin{block}{Extending proof methods}
        To prove a disjunction $p \vee q$, we may choose to prove $\neg p \implies q$.
    \end{block}
    \begin{example}
        Prove that if $a,b \in \mathbb{Z}$ and $a \mid b$, then $a$ is odd or $b$ is even. 
    \end{example}
\end{frame}

\begin{frame}[t]{Divisibility}
    \begin{example}
        Prove that if $a,b \in \mathbb{Z} \setminus \{0\}$, $a \mid b$, and $a + b$ is odd, then $a$ is odd. 
    \end{example}
\end{frame}

\begin{frame}[t]{Modular Arithmetic}
    \begin{example}
        Show that if $a,d \in \mathbb{Z^+}$, then $(-a) \textbf{ div } d = -a \textbf{ div } d$ iff $d \mid a$.
    \end{example}
\end{frame}

\begin{frame}[t]{Modular Arithmetic}
    \begin{theorem}[The Division Algorithm]
        Let $a,b \in \mathbb{Z}$ with $b > 0$. Then there exists unique $q,r \in \mathbb{Z}$ such that $a = bq+r$, where $0 \leq r < b$. 
    \end{theorem}
    \begin{example}
        Show that if $n,k \in \mathbb{Z^+}$, then $\lceil n/k \rceil = \lfloor (n-1)/k \rfloor + 1$
    \end{example}
\end{frame}

\begin{frame}{Modular Arithmetic}
    \begin{example}
        Evaluate these quantities. \begin{enumerate}
            \item $-17 \textbf{ mod } 2$ \vspace{8mm}
            \item $144 \mod 7$ \vspace{8mm}
            \item $-101 \mod 13$ \vspace{8mm}
            \item $199 \mod 19$ \vspace{10mm}
        \end{enumerate}
    \end{example}
\end{frame}

\begin{frame}[t]{Congruences}
    \begin{definition}
        Let $a,b \in \mathbb{Z}$, and let $m \in \mathbb{Z^+}$. The integers $a$ and $b$ are \textbf{congruent modulo} m, written $a \equiv b \mod m$, if $m \mid (a - b)$. 
    \end{definition}
    \begin{example}
        Show that if $a \equiv b \, (\text{mod } m)$ and $c \equiv d \, (\text{mod } m)$, where $a,b,c,d,m \in \mathbb{Z}$ with $m \geq 2$, then $a - c \equiv b - d \, (\text{mod } m)$.
    \end{example}
\end{frame}

\end{document}
