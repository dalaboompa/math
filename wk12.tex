\documentclass{beamer}
\usetheme{Boadilla}
\usepackage{algorithm,algorithmic}
\renewcommand{\mod}{\text{ mod }}
\renewcommand{\div}{\textbf{ div }}
\renewcommand{\implies}{\rightarrow}
\let\cd\cdot
%Information to be included in the title page:
\title{COMPSCI 1DM3 Tut 4}
\subtitle{Week 12}
\author{Zitong Gu}
\institute{McMaster University}
\date{Winter 2023}

\begin{document}

\frame{\titlepage}

\begin{frame}[t]{Counting}
    \begin{example}
        How many bit strings of length ten both begin and end with a $1$?
    \end{example}
\end{frame}

\begin{frame}[t]{Counting}
    \begin{example}
        How many strings are there of lowercase letters of length four or less, not counting the empty string?
    \end{example}
\end{frame}

\begin{frame}[t]{Counting}
    \begin{example}
        How many strings are there of four lowercase letters that have the letter $x$ in them?
    \end{example}
\end{frame}

\begin{frame}[t]{Counting}
    \begin{example}
        How many functions are there from the set $\{1,2,\dots,n\}$, where $n$ is a positive integer, to the set $\{0,1\}$?
    \end{example}
\end{frame}

\begin{frame}[t]{Counting}
    \begin{example}
        How many functions are there from the set $\{1,2,\dots,n\}$, where $n$ is a positive integer, to the set $\{0,1\}$ \begin{enumerate}
            \item that are one-to-one?
            \item that assign $0$ to both $1$ and $n$?
            \item that assign $1$ to exactly one of the positive integers less than $n$?
        \end{enumerate}
    \end{example}
\end{frame}

\begin{frame}[t]{Counting}
    \begin{example}
        A \textbf{palindrome} is a string whose reversal is identical to the string. How many bit strings of length $n$ are palindromes?
    \end{example}
\end{frame}

\begin{frame}[t]{Counting}
    \begin{theorem}[Inclusion-exclusion Principle]
        $|A \cup B| = |A| + |B| - |A \cap B|$
    \end{theorem}
    \begin{example}
        How many bit strings of length $10$ either begin with three $0$s or end with two $0$s?
    \end{example}
\end{frame}

\begin{frame}[t]{Counting}
    \begin{example}
        How many bit strings of length $10$ contain either five consecutive $0$s or five consecutive $1$s?
    \end{example}
\end{frame}

\begin{frame}[t]{Counting}
    \begin{theorem}[The Product Rule]
        Suppose that a procedure can be broken down into a sequence of two tasks. If there are $n_1$ ways to do the first task and for each of these ways of doing the first task, there are $n_2$ ways to do the second task, then there are $n_1n_2$ ways to do the procedure.
    \end{theorem}
    \begin{example}
        Use the product rule to show that there are $2^{2^n}$ different truth tables for propositions in $n$ variables.
    \end{example}
\end{frame}

\begin{frame}[t]{The Pigenhole Principle}
    \begin{example}
        How many numbers must be selected from the set $\{1, 2, 3, 4, 5, 6\}$ to guarantee that at least one pair of these numbers add up to $7$?
    \end{example}
\end{frame}

\begin{frame}[t]{The Pigenhole Principle}
    \begin{example}
        There are $38$ different time periods during which classes at a university can be scheduled. If there are $677$ different classes, how many different rooms will be needed?
    \end{example}
\end{frame}

\begin{frame}[t]{The Pigenhole Principle}
    \begin{example}
        Show that if $f$ is a function from $S$ to $T$, where $S$ and $T$ are nonempty finite sets and $m = \lceil |S|/|T| \rceil$, then there are at least $m$ elements of $S$ mapped to the same value of $T$. That is, show that there are distinct elements $s_1,s_2,\dots,s_m$ of $S$ such that $f(s_1) = f(s_2) = \dots = f(s_m)$.
    \end{example}
\end{frame}

\begin{frame}[t]{The Pigenhole Principle}
    \begin{example}
        Let $n_1, n_2,\dots, n_t$ be positive integers. Show that if $n_1 + n_2 +\dots+ n_t - t + 1$ objects are placed into $t$ boxes, then for some $i$, where $i = 1, 2,\dots, t$, the $i$th box contains at least $n_i$ objects.
    \end{example}
\end{frame}

\end{document}
