\documentclass{beamer}
\usetheme{Boadilla}
\usepackage{algorithm,algorithmic}
\renewcommand{\mod}{\textbf{ mod }}
\renewcommand{\div}{\textbf{ div }}
\renewcommand{\implies}{\rightarrow}
%Information to be included in the title page:
\title{COMPSCI 1DM3 Tut 4}
\subtitle{Week 09}
\author{Zitong Gu}
\institute{McMaster University}
\date{Winter 2023}

\begin{document}

\frame{\titlepage}

\begin{frame}[t]{Modular Arithmetic}
    \begin{example}
        Prove that if $a$ is an integer that is not divisible by $3$, then $(a+1)(a+2)$ is divisible by $3$.
    \end{example}
\end{frame}

\begin{frame}[t]{Modular Arithmetic}
    \begin{example}
        Prove that if $a$ is a positive integer, then $4$ does not divide $a^2+2$.
    \end{example}
\end{frame}

\begin{frame}[t]{Modular Arithmetic}
    \begin{example}
        Find each of these values. \begin{enumerate}
            \item $(99^2 \mod 32)^3 \mod 15$
            \item $(3^4 \mod 17)^2 \mod 11$
            \item $(19^3 \mod 23)^2 \mod 31$
            \item $(89^3 \mod 79)^4 \mod 26$
        \end{enumerate}
    \end{example}
\end{frame}

\begin{frame}{Modular Arithmetic}
    \begin{example}
        Find counterexamples to each of these statements about congruences. 
        \begin{enumerate}
            \item If $ac \equiv bc \mod m$, where $a,b,c,m \in \mathbb{Z}$ with $m \geq 2$, then $a \equiv b \mod m$.
            \vspace{16mm}
            \item If $a \equiv b \mod m$ and $c \equiv d \mod m$, where $a,b,c,d,m \in \mathbb{Z}$ with $c,d$ positive and $m \geq 2$, then $a^c \equiv b^d \mod m$.
            \vspace{18mm}
        \end{enumerate}
    \end{example}
\end{frame}

\begin{frame}[t]{Modular Arithmetic}
    \begin{example}
        Show that if $n$ is an integer then $n^2 \equiv 0$ or $1 \, (\textbf{mod } 4)$. 
    \end{example}
\end{frame}

\begin{frame}[t]{Modular Arithmetic}
    \begin{example}
        Show that if $m$ is a positive integer of the form $4k+3$ for some nonnegative integer $k$, then $m$ is not the sum of the squares of two integers. 
    \end{example}
\end{frame}

\begin{frame}[t]{Modular Arithmetic}
    \begin{example}
        Show that if $a,b,k,m \in \mathbb{Z}$ such that $k \geq 1, m\geq 2$, and $a \equiv b \mod m$, then $a^k \equiv b^k \mod m$.
    \end{example}
\end{frame}

\begin{frame}[t]{Modular Arithmetic}
    \begin{example}
        Determine whether each of the functions $f(a) = a \div d$ and $g(a) = a \mod d$, where $d$ is a fixed positive integer, from the set of integers to the set of integers, is one-to-one, and determine whether each of these functions is onto. 
    \end{example}
\end{frame}

\begin{frame}[t]{Integer Representations}
    \begin{example}
        Convert the binary expansion of each of these integers to a decimal expansion. \begin{enumerate}
            \item $(1 \, 1111)_2$
            \item $(10 \, 1011 \, 0101)_2$
            \item $(11 \, 1011 \, 1110)_2$
            \item $(111 \, 1100 \, 0001 \, 1111)_2$
        \end{enumerate}
    \end{example}
\end{frame}

\begin{frame}[t]{Integer Representations}
    \begin{example}
        Convert the octal expansion of each of these integers to a binary expansion. \begin{enumerate}
            \item $(572)_8$
            \item $(1604)_8$
            \item $(423)_8$
            \item $(2417)_8$
        \end{enumerate}
    \end{example}
\end{frame}

\begin{frame}[t]{Integer Representations}
    \begin{example}
        Give a procedure of converting from the hexadecimal expansion of an integer to its octal expansion using binary notation as an intermediate step. 
    \end{example}
\end{frame}

\begin{frame}[t]{Integer Representations}
    \begin{example}
        Find the sum and the product of each these pairs of numbers. Express your answer as a binary expansion. \begin{enumerate}
            \item $(100 \, 0111)_2, (111 \, 0111)_2$
            \item $(1110 \, 1111)_2, (1011 \, 1101)_2$
            \item $(10 \, 1010 1010)_2, (1 \, 1111 \, 0000)_2$
            \item $(10 \, 0000 \, 0001)_2, (11 \, 1111 \, 1111)_2$
        \end{enumerate}
    \end{example}
\end{frame}

\begin{frame}[t]{Integer Representations}
    \begin{example}
        show that a positive integer is divisible by $3$ iff the sum of its decimal digits is divisible by $3$.
    \end{example}
\end{frame}

\begin{frame}[t]{Integer Representations}
    \begin{example}
        Suppose that $n$ and $b$ are positive integers with $b \geq 2$ and the base $b$ expansion of $n$ is $n = (a_ma_{m-1}\dots a_1a_0)_b$. Find the base $b$ expansion of \begin{enumerate}
            \item $bn$
            \item $b^2n$
            \item $\lfloor n/b \rfloor$
            \item $\lfloor n/b^2 \rfloor$
        \end{enumerate}
    \end{example}
\end{frame}

\begin{frame}[t]{Integer Representations}
    \begin{example}
        Prove that if $n$ and $b$ are positive integers with $b \geq 2$, then the base $b$ representation of $n$ has $\lfloor \log_{b}{n} \rfloor + 1$ digits.
    \end{example}
\end{frame}

\end{document}
