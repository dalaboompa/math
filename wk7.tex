\documentclass{beamer}
\usetheme{Boadilla}
%Information to be included in the title page:
\title{COMPSCI 1DM3 Tut 4}
\subtitle{Week 07}
\author{Zitong Gu}
\institute{McMaster University}
\date{Winter 2023}

\begin{document}

\frame{\titlepage}

\begin{frame}{Big-\textit{O}}
    % introduce the definition of O(g)
    \begin{definition}
        For a given function $g(x)$, we denote by $O(g(x))$ the \textbf{set of functions}
        $$
            O(g(x)) = \{ f(x): (\exists c > 0)(\exists x_0 \geq 0)[(\forall x \geq x_0)[0 \leq f(x) \leq c \cdot g(x)]] \}
        $$
    \end{definition}
    % meaning of O
\end{frame}

\begin{frame}{Notational Abuse}
    \begin{itemize}
        \item \textbf{Equal Sign.} $O(g(x))$ is a set of functions, but computer scientists often write $f(x) = O(g(x))$ instead of $f(x) \in O(g(x))$.
        \begin{example}
            Consider $f(x) = x^2$ and $g(x) = 2x^2$. \vspace{12mm}
        \end{example}
        \item \textbf{Domain.} The domain of $f(x)$ is typically the set of natural numbers $\{0, 1, 2, \dots\}$. We may restrict the domain to a subset of $\mathbb{N}$ or extend to $\mathbb{R}$. 
        \item \textbf{Non-negative functions.} When using Big-Oh notation, we assume that the involved functions are (asymptotically) non-negative. 
    \end{itemize}
\end{frame}

\begin{frame}[t]{Big-\textit{O}}
    \begin{example}
        Determine whether each of these functions is $\textit{O}(x)$.
        \begin{enumerate}
            \item $f(x) = 10$
            \vspace{10mm}
            \item $f(x) = 3x + 7$
            \vspace{10mm}
            \item $f(x) = x^2 + x + 1$
            \vspace{10mm}
            \item $f(x) = 5 \log{x}$
            \vspace{10mm}
        \end{enumerate}
    \end{example}
\end{frame}

\begin{frame}[t]{Big-\textit{O}}
    \begin{example}
        Determine whether each of these functions is $\textit{O}(x)$.
        \begin{enumerate}
            \setcounter{enumi}{4}
            \item $f(x) = \lfloor x \rfloor$
            \vspace{20mm}
            \item $f(x) = \lceil x/2 \rceil$
            \vspace{22mm}
        \end{enumerate}
    \end{example}
\end{frame}

\begin{frame}[t]{Big-\textit{O}}
    \begin{theorem}
        Let $f(x) = c_nx^n + c_{n-1}x^{n-1} + \dots + c_1x + c_0$, where $c_0, c_1, \dots, c_n \in \mathbb{R}$. Then $f(x) \in O(x^n)$.  
    \end{theorem}
    \begin{example}
        Determine whether each of these functions is $O(x^2)$.
        \begin{enumerate}
            \item $f(x) = 17x + 11$
            \item $f(x) = x^2 + 1000$
            \item $f(x) = x^4/2$
        \end{enumerate}
    \end{example}
\end{frame}

\begin{frame}[t]{Big-\textit{O}}
    \begin{example}
        Determine whether each of these functions is $O(x^2)$.
        \begin{enumerate}
            \item $f(x) = x\log{x}$
            \vspace{12mm}
            \item $f(x) = 2^x$
            \vspace{12mm}
            \item $f(x) = \lfloor x \rfloor \cdot \lceil x \rceil$
            \vspace{14mm}
        \end{enumerate}
    \end{example}
\end{frame}

\begin{frame}[t]{Big-\textit{O}}
    \begin{example}
        Show that the function $f(x) = \dfrac{x^2 + 1}{x + 1}$ is $O(x)$.
    \end{example}
\end{frame}

\begin{frame}[t]{Big-\textit{O}}
    \begin{example}
        Show that the function $\dfrac{x^3 + 2x}{2x + 1}$ is $O(x^2)$.
    \end{example}
\end{frame}

\begin{frame}[t]{Big-\textit{O}}
    \begin{example}
        Find the least integer $n$ such that $f(x)$ is $O(x^n)$ for each of these functions. 
        \begin{enumerate}
            \item $f(x) = 2x^3 + x^2\log{x}$
            \item $f(x) = 3x^3 + (\log{x})^4$
            \item $f(x) = \dfrac{x^4 + x^2 + 1}{x^3 + 1}$
            \item $f(x) = \dfrac{x^4 + 5\log{x}}{x^4 + 1}$
        \end{enumerate}
    \end{example}
\end{frame}

\begin{frame}[t]{Big-\textit{O}}
    \begin{example}
        Explain what it means for a function to be $O(1)$.
    \end{example}
\end{frame}

\begin{frame}[t]{Big-\textit{O}}
   \begin{example}
        Let $k \in \mathbb{Z^+}$. Show that $1^k + 2^k + \dots + n^k$ is $O(n^{k+1})$
   \end{example}
\end{frame}

\begin{frame}{}
    \begin{center}
        The End
    \end{center}
\end{frame}

\end{document}
