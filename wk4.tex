\documentclass{beamer}
\usetheme{Boadilla}
%Information to be included in the title page:
\title{COMPSCI 1DM3 Tut 4}
\subtitle{Week 04}
\author{Zitong Gu}
\institute{McMaster University}
\date{Winter 2023}

\begin{document}

\frame{\titlepage}

\begin{frame}{Announcement}
    \begin{enumerate}
        \item To get participation marks, you must ask math-related questions or answer my questions (I don't always have questions for you).
        \item Notes will be posted on MS Teams (join code: eu0mvpu) after each tutorial. 
    \end{enumerate}
\end{frame}

\begin{frame}{Set Membership}
    \begin{example}
        For each of the following sets, determine whether 2 is an element of that set.
        \begin{enumerate}
            \item $\{ x \in \mathbb{R} \mid x \text{ is an integer greater than } 1\}$
            \item $\{ x \in \mathbb{R} \mid x \text{ is the square of an integer}\}$
            \item \{2, \{2\}\}
            \item \{\{2\},\{\{2\}\}\}
            \item \{\{2\},\{2,\{2\}\}\}
            \item \{\{\{2\}\}\}
        \end{enumerate}
    \end{example}
\end{frame}

\begin{frame}[t]{Subset}
    \begin{definition}
        Let $A$ and $B$ be sets. $A \subseteq B$ if $(\forall x)(x \in A \Rightarrow x \in B)$.
    \end{definition}
    \begin{definition}
        Let $A$ and $B$ be sets. $A \subset B$ if $A \subseteq B$ and $A \neq B$.
    \end{definition}
    \begin{theorem}
        For every set $S$, \begin{enumerate}
            \item $\emptyset \subseteq S$
            \item $S \subseteq S$
        \end{enumerate}
    \end{theorem}
\end{frame}

\begin{frame}{Subset}
    %State the definition of A subsets B
    %State the def of proper subset
    %State two theorems: for all set S, emptyset seubsets S; S subsets itself.
    \begin{example}
        Determine whether each of these statements is true or false.
        \begin{enumerate}
            \item $x \in \{x\}$
            \item $\{x\} \subseteq \{x\}$
            \item $\{x\} \subset \{x\}$
            \item $\{x\} \in \{x\}$
            \item $\{x\} \in \{\{x\}\}$
            \item $\emptyset \subseteq \{x\}$
            \item $\emptyset \in \{x\}$
            \item $\emptyset \subseteq \emptyset$
            \item $\emptyset \in \{\emptyset\}$
        \end{enumerate}
    \end{example}
\end{frame}

\begin{frame}{The notation of $|$  $|$}
    %introduce the cardinality of a set, the norm of a vector in R^n
\end{frame}

\begin{frame}[t]{Sets - Cartesian Products}
    \begin{example}
        Find $A^2 = A \times A$ if (a) $A = \{0,1,3\}$, (b) $A = \{1,2,a,b\}$.
    \end{example}
\end{frame}

\begin{frame}[t]{Sets - Cartesian Products}
    \begin{example}
        Let $A = \{a,b,c\}, B = \{x,y\}, \text{ and } C = \{0,1\}$. Find 
        \begin{enumerate}
            \item $A \times B \times C$
            \item $C \times B \times A$
            \item $C \times A \times B$
            \item $B^3$
        \end{enumerate}
    \end{example}
\end{frame}

\begin{frame}{Set Operations}
    %Left for definitions of set union, intersection, complement, and difference.
\end{frame}

\begin{frame}[t]{More on Subsets}
    \begin{example}
        Let $A,B, \text{ and } C$ be sets. Show that \begin{enumerate}
            \item $(A \cup B) \subseteq (A \cup B \cup C)$
            \item $(A \cap B \cap C) \subseteq (A \cap B)$
            \item $(A - B) - C \subseteq A - C$
        \end{enumerate}
    \end{example}
\end{frame}

\begin{frame}{Set Identities}
    We will demonstrate two methods for proving set identities: \begin{enumerate}
        \item[1] Subset Method
        \item[2] Transformations by existing identities
    \end{enumerate}
\end{frame}

\begin{frame}[t]{Set Identities}
    \begin{theorem}
        Let $A$ and $B$ be subsets of a universe, then $$A - B = A \cap \overline{B}$$
    \end{theorem}
\end{frame}

\begin{frame}{}
    %left blank
\end{frame}

\begin{frame}[t]{Set Identities}
    %Give the alternative def of set difference A - B = A insersects not B
    \begin{example}
        Let $A,B, \text{ and } C$ be sets. Show that \begin{enumerate}
            \item $(A - C) \cap (C - B) = \emptyset$
            \item $(B - A) \cup (C - A) = (B \cup C) - A$
        \end{enumerate}
    \end{example}
\end{frame}

\begin{frame}[t]{Set Identities}
    \begin{example}
        Let $A,B, \text{ and } C$ be sets. Show that $(A - B) - C = (A - C) - (B - C)$.
    \end{example}
\end{frame}

\begin{frame}[t]{Venn Diagram}
    \begin{example}
        Draw the Venn diagrams for each of these combinations of the sets $A, B, \text{ and } C$. 
        \begin{enumerate}
            \item $A \cap (B - C)$
            \item $(A \cap B) \cup (A \cap C)$
            \item $(A \cap \overline{B}) \cup (A \cap \overline{C})$
        \end{enumerate}
    \end{example}
\end{frame}

\begin{frame}[t]{Venn Diagram}
    \begin{example}
        What can you say about set $A$ and set $B$, where $A,B \subseteq \mathcal{U}$, if we know that \begin{enumerate}
            \item $A \cup B = A$
            \item $A \cap B = A$
            \item $A - B = A$
            \item $A \cap B = B \cap A$
            \item $A - B = B - A$
        \end{enumerate}
    \end{example}
\end{frame}

\begin{frame}{}
    %left blank
\end{frame}

\begin{frame}[t]{Generalized Unions and Intersections}
    \begin{example}
        Let $A_i = \{\dots,-2,-1,0,1,2,\dots,i\}$. Find $\bigcup_{i=1}^{n} A_i$ and $\bigcap_{i=1}^{n} A_i$.
    \end{example}
\end{frame}

\begin{frame}[t]{Power Set}
    \begin{definition}
        Given a set $S$, the \textbf{power set} of $S$ is the set of all subsets of $S$ denoted by $\mathcal{P}(S)$.
    \end{definition}
    \begin{theorem}
        Let $A$ and $B$ be sets, then $$\mathcal{P}(A) \cup \mathcal{P}(B) \subseteq \mathcal{P}(A \cup B)$$
    \end{theorem}
\end{frame}

\begin{frame}{}
    %left blank
\end{frame}

\begin{frame}[t]{Power Set}
    \begin{example}
        Determine $|\mathcal{P}(\mathcal{P}(\mathcal{P}(\emptyset)))|$.
    \end{example}
\end{frame}

\end{document}